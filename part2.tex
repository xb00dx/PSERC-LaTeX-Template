\documentclass[letterpaper, 12pt, oneside]{book}
\usepackage[top=1in, bottom=1in,margin=1in]{geometry}
\usepackage{lscape}
\usepackage{setspace}
\usepackage{longtable}
\usepackage{amsmath,amsthm,amsfonts, amssymb}
% \usepackage{mathptmx} % times new roman 
\usepackage{newtxtext,newtxmath}
% \usepackage{newtxtext,newtxmath}
% For python environments
% \usepackage{listings}
% \usepackage{fontspec}
% \usepackage{minted}
\usepackage{cite}
\usepackage{wasysym}
\usepackage{textcomp}
\usepackage{comment}
\usepackage{titlesec}
\usepackage{tocloft}

% \usepackage{fontspec}
% \setmainfont{Times New Roman}

%\titleformat{\chapter}[display]
%  {\normalfont\bfseries\centering}{}{0pt}{\Large\thechapter. }

%\titleformat{\chapter}{\Large\bfseries\centering}{\thechapter. }{1em}{}
% \titleformat{\chapter}{\large\bfseries\centering}{\thechapter.}{1em}{}
\titleformat{\chapter}{\large\bfseries}{\thechapter.}{1em}{}
[\vspace{-3ex}
\rule{\textwidth}{1pt}]
\titlespacing\chapter{0pt}{12pt plus 4pt minus 2pt}{0pt plus 2pt minus 2pt}

\titleformat{\section}{\normalsize\bfseries}{\thesection.}{1em}{}
\titlespacing\section{0pt}{12pt plus 4pt minus 2pt}{0pt plus 2pt minus 2pt}

\titleformat{\subsection}{\normalsize\bfseries}{\thesubsection.}{1em}{}
\titlespacing\subsection{0pt}{12pt plus 4pt minus 2pt}{0pt plus 2pt minus 2pt}



% \titleformat{\tableofcontents}{\large}{}{}{}
% \titlespacing{\tableofcontents}{0pt}{0pt}{\baselineskip}
% \titleformat{\listoffigures}{\large}{}{}{}
% \titlespacing{\listoffigures}{0pt}{0pt}{\baselineskip}
% \titleformat{\listoftables}{\large\bfseries}{}{}{}
% \titlespacing{\listoftables}{0pt}{0pt}{\baselineskip}


\usepackage{subfig}

\usepackage{hyperref}
\hypersetup{
    colorlinks=true,
    linkcolor=black,
    filecolor=magenta,      
    urlcolor=cyan,
}
 
\urlstyle{same}

% \setsansfont{Calibri}
% \setmonofont{Consolas}

\usepackage[toc,page]{appendix}

% \usepackage{subcaption}
\usepackage{graphicx}
\graphicspath{{./fig/}}

\theoremstyle{plain}
\newtheorem{thm}{Theorem}
\newtheorem{lem}[thm]{Lemma}
\newtheorem{prop}[thm]{Proposition}
\newtheorem*{cor}{Corollary}

\theoremstyle{definition}
\newtheorem{defn}{Definition}
\newtheorem{conj}{Conjecture}
\newtheorem{exmp}{Example}

\theoremstyle{remark}
\newtheorem*{rem}{Remark}
\newtheorem*{note}{Note}

\newcommand{\se}[1]{{\textcolor{black}{{#1}}}}
\newcommand{\minimize}{\mathop{\textup{minimize}}}
\newcommand{\maximize}{\mathop{\textup{maximize}}}


\usepackage{algorithmic}
% \usepackage{graphicx}
\usepackage{textcomp}
\usepackage{caption}
\usepackage{wasysym}
\usepackage{float}
% \usepackage[sort,numbers]{natbib}
\usepackage{multirow}
\usepackage{nomencl}
\usepackage{textcomp}
\usepackage{array}
\usepackage{wrapfig}
\usepackage{tabularx}
\usepackage{etoolbox}
\renewcommand\nomgroup[1]{%
  \item[\bfseries
  \ifstrequal{#1}{A}{Abbreviations}{%
  \ifstrequal{#1}{B}{Terminology Definitions}{%
  \ifstrequal{#1}{C}{Symbols}{}}}%
]}
\setlength{\nomitemsep}{-\parsep}
\setlength\parindent{0pt}
\newcommand{\nomunit}[1]{%
\renewcommand{\nomentryend}{\hspace*{\fill}#1}}
\setlength{\parskip}{1em}
\pagestyle{plain}

% \makenomenclature
%\makeindex cas-dc-template-v4.nlo -s nomencl.ist -o cas-dc-template-v4.nls

%%%Author definitions
\def\tsc#1{\csdef{#1}{\textsc{\lowercase{#1}}\xspace}}
\tsc{WGM}
\tsc{QE}
\tsc{EP}
\tsc{PMS}
\tsc{BEC}
\tsc{DE}

\begin{document}
\pagenumbering{roman}


\begin{titlepage}
\centering
\quad \\
\quad \\
\quad \\
\quad \\
\quad \\
\quad \\
%\vspace{70pt}
% \begin{LARGE}
% \begin{spacing}{1}
\begingroup
    \fontsize{18pt}{20pt}\selectfont
    \textbf{Part 2} \\
    \quad \\
    \quad \\
    \textbf{Enhanced Transmission and Distribution System Operation and Planning with Improved Dynamic Load Modeling}
\endgroup
% \end{spacing}
% \end{LARGE}

\vspace{80pt}

\begingroup
\fontsize{16pt}{20pt}\selectfont
Le Xie\\
Xinbo Geng, Graduate Student\\
Athindra Venkatraman, Graduate Student\\
\quad \\
Texas A\&M University    
\endgroup


\end{titlepage}

\newpage
\renewcommand{\contentsname}{\hspace{6.5cm}Table of Contents} % \hspace{6cm} is to make the title in the center
\renewcommand{\listtablename}{\hspace{7cm}List of Tables} % \hspace{6cm} is to make the title in the center
\renewcommand{\listfigurename}{\hspace{7cm}List of Figures} % \hspace{6cm} is to make the title in the center
\renewcommand{\cftloftitlefont}{\bfseries\large}
\renewcommand{\cfttoctitlefont}{\bfseries\large}
\renewcommand{\cftlottitlefont}{\bfseries\large}
% \renewcommand{\cftpartfont}{\normalfont\sffamily\bfseries}% \part font in ToC
\renewcommand{\cftchapfont}{\normalfont\large}    % \chapter font in ToC
\renewcommand{\cftsecfont}{\normalfont}           % \section font in ToC
\renewcommand{\cftsubsecfont}{\normalfont}        % \subsection font in ToC
\renewcommand{\cftsubsubsecfont}{\normalfont\small}       % \subsubsection font in ToC


% \begin{comment}
% \begin{large}
% \begin{center}
% \textbf{Table of Contents} 
% \end{center}
% \end{large}

% \makeatletter
% \renewcommand\tableofcontents{%
%     \@starttoc{toc}%
% }
% \makeatother
% \end{comment}

\newpage
\quad \vspace{-1in} % to make the title "Table of Contents" at top 1in margin
\tableofcontents

%\renewcommand{\cftafterloftitle{\newline}}
\newpage
\quad \vspace{-1in} % to make the title "List of Figures" at top 1in margin
\listoffigures

\newpage
\quad \vspace{-1in} % to make the title "List of Tables" at top 1in margin
\listoftables

% \begin{minted}[mathescape,
%                linenos,
%                numbersep=5pt,
%                gobble=2,
%                frame=lines,
%                framesep=2mm]{csharp}
%   string title = "This is a Unicode π in the sky"
%   /*
%   Defined as $\pi=\lim_{n\to\infty}\frac{P_n}{d}$ where $P$ is the perimeter
%   of an $n$-sided regular polygon circumscribing a
%   circle of diameter $d$.
%   */
%   const double pi = 3.1415926535
% \end{minted}

% \end{abstract}


\newpage
\pagenumbering{arabic}



\chapter[Short Chapter Title]{This is the full title of a Chapter}
\label{chap:chapter1}

\section{Introduction} % (fold)
\label{sec:introduction}

\subsection{Background 1} % (fold)
\label{sub:background_1}

% subsection background_1 (end)

\subsection{Background 2} % (fold)
\label{sub:background_2}

% subsection background_2 (end)
% section introduction (end)




\renewcommand\bibname{References}
\bibliographystyle{IEEEtran}
\bibliography{references}
% \bibliographystyle{plain}
% \bibliography{myreferences}

\end{document}



